% A LaTeX template for MSc Thesis submissions to 
% Politecnico di Milano (PoliMi) - School of Industrial and Information Engineering
% 
% People who have contributed to this document:
% Florian Daniel, S. Bonetti, A. Gruttadauria, G. Mescolini, A. Zingaro, and D. Piantella.
%
% Copyright (C) 2022 - CC BY-NC 4.0
% This work is licensed under the Creative Commons Attribution-NonCommercial 4.0 International License. 
% To view a copy of this license, visit http://creativecommons.org/licenses/by-nc/4.0/ 
% or send a letter to Creative Commons, PO Box 1866, Mountain View, CA 94042, USA.
%

% DO NOT REMOVE THE FOLLOWING LINE
\documentclass{Configuration_Files/PoliMi3i_thesis}

%------------------------------------------------------------------------------
%	REQUIRED PACKAGES AND  CONFIGURATIONS
%------------------------------------------------------------------------------

% CONFIGURATIONS
\usepackage{parskip} % For paragraph layout
\usepackage{setspace} % For using single or double spacing
\usepackage{emptypage} % To insert empty pages
\usepackage{multicol} % To write in multiple columns (executive summary)
\setlength\columnsep{15pt} % Column separation in executive summary
\setlength\parindent{0pt} % Indentation
\raggedbottom  

% PACKAGES FOR TITLES
\usepackage{titlesec}
% \titlespacing{\section}{left spacing}{before spacing}{after spacing}
\titlespacing{\section}{0pt}{3.3ex}{2ex}
\titlespacing{\subsection}{0pt}{3.3ex}{1.65ex}
\titlespacing{\subsubsection}{0pt}{3.3ex}{1ex}
\usepackage{color}

% PACKAGES FOR LANGUAGE AND FONT
\usepackage[english]{babel} % The document is in English  
\usepackage[utf8]{inputenc} % UTF8 encoding
\usepackage[T1]{fontenc} % Font encoding
\usepackage[11pt]{moresize} % Big fonts

% PACKAGES FOR IMAGES
\usepackage{graphicx}
\usepackage{transparent} % Enables transparent images
\usepackage{eso-pic} % For the background picture on the title page
\usepackage{subfig} % Numbered and caption subfigures using \subfloat.
\usepackage{tikz} % A package for high-quality hand-made figures.
\usetikzlibrary{}
\graphicspath{{./Images/}} % Directory of the images
\usepackage{caption} % Coloured captions
\usepackage{xcolor} % Coloured captions
\usepackage{amsthm,thmtools,xcolor} % Coloured "Theorem"
\usepackage{float}

% STANDARD MATH PACKAGES
\usepackage{amsmath}
\usepackage{amsthm}
\usepackage{amssymb}
\usepackage{amsfonts}
\usepackage{bm}
\usepackage[overload]{empheq} % For braced-style systems of equations.
\usepackage{fix-cm} % To override original LaTeX restrictions on sizes

% PACKAGES FOR TABLES
\usepackage{tabularx}
\usepackage{longtable} % Tables that can span several pages
\usepackage{colortbl}

% PACKAGES FOR ALGORITHMS (PSEUDO-CODE)
\usepackage{algorithm}
\usepackage{algorithmic}

% PACKAGES FOR REFERENCES & BIBLIOGRAPHY
\usepackage[colorlinks=true,linkcolor=black,anchorcolor=black,citecolor=black,filecolor=black,menucolor=black,runcolor=black,urlcolor=black]{hyperref} % Adds clickable links at references
\usepackage{cleveref}
\usepackage[square, numbers, sort&compress]{natbib} % Square brackets, citing references with numbers, citations sorted by appearance in the text and compressed
\bibliographystyle{abbrvnat} % You may use a different style adapted to your field

% OTHER PACKAGES
\usepackage{pdfpages} % To include a pdf file
\usepackage{afterpage}
\usepackage{lipsum} % DUMMY PACKAGE
\usepackage{fancyhdr} % For the headers
\fancyhf{}


%----------------------------------------------------------------------------
%	PACKAGES AND COMMANDS USED FOR THIS GUIDE.
%   YOU CAN DELETE THIS LINE WHEN WRITING YOUR DOCUMENT
%----------------------------------------------------------------------------
\usepackage{framed}
\usepackage{amsmath,amssymb}   
\newcommand{\note}[2]{\begin{framed}\sffamily\small
\noindent \textbf{#1} \\ #2\end{framed}}
\usepackage{bbding}

% Input of configuration file. Do not change config.tex file unless you really know what you are doing. 
\input{Configuration_Files/config}

%----------------------------------------------------------------------------
%	NEW COMMANDS DEFINED
%----------------------------------------------------------------------------

% EXAMPLES OF NEW COMMANDS
\newcommand{\bea}{\begin{eqnarray}} % Shortcut for equation arrays
\newcommand{\eea}{\end{eqnarray}}
\newcommand{\e}[1]{\times 10^{#1}}  % Powers of 10 notation

%----------------------------------------------------------------------------
%	ADD YOUR PACKAGES (be careful of package interaction)
%----------------------------------------------------------------------------

%----------------------------------------------------------------------------
%	ADD YOUR DEFINITIONS AND COMMANDS (be careful of existing commands)
%----------------------------------------------------------------------------

%----------------------------------------------------------------------------
%	BEGIN OF YOUR DOCUMENT
%----------------------------------------------------------------------------

\begin{document}

\fancypagestyle{plain}{%
\fancyhf{} % Clear all header and footer fields
\fancyhead[RO,RE]{\thepage} %RO=right odd, RE=right even
\renewcommand{\headrulewidth}{0pt}
\renewcommand{\footrulewidth}{0pt}}

%----------------------------------------------------------------------------
%	TITLE PAGE
%----------------------------------------------------------------------------

\pagestyle{empty} % No page numbers
\frontmatter % Use roman page numbering style (i, ii, iii, iv...) for the preamble pages

\puttitle{
	title=Title, % Title of the thesis
	name=Name Surname, % Author Name and Surname
	course=Xxxxxxx Engineering - Ingegneria Xxxxxxx, % Study Programme (in Italian)
	ID  = 000000 (Matricola),  % Student ID number (numero di matricola)
	advisor= Prof. Name Surname, % Supervisor name
	coadvisor={Name Surname, Name Surname}, % Co-Supervisor name, remove this line if there is none
	academicyear={20XX-XX},  % Academic Year
} % These info will be put into your Title page 


\section*{About this template}
At Politecnico di Milano, a huge number of former Compute Science and Engineering students (including myself) has written their master thesis starting from a very useful and pleasant \LaTeX~template, written by Prof. Florian Daniel\footnote{\url{https://www.deib.polimi.it/eng/people/details/165694}}, who passed away prematurely in 2020.

In 2021, PoliMi published the first official template\footnote{created by S. Bonetti, A. Gruttadauria, G. Mescolini, and A. Zingaro. Available at\\ \url{https://www.ingindinf.polimi.it/en/1/teaching/lectures-and-exams/degree-examinations}} for master theses and I am sure if Florian were still here, he would quickly update his original template to follow the new guidelines. 

I tried my best to merge the two documents, preserving the original style and taste as much as possible.
Most of the texts and comments come from Florian's template, while the structure and graphics of the document are the official ones.

You can still find the original template, together with other helpful and inspiring material, on Florian's website \url{https://www.floriandaniel.it/}.

\hfill Davide Piantella

\hfill July, 2022

\bigskip

\note{About this template}{With this template I want to give you some input on how to structure your thesis if you develop your thesis with me in Politecnico di Milano. Next to the pure structure, which you should reuse and adapt to your own needs, the document also contains instructions on how to approach the different sections, the writing and, sometimes, even the work on your thesis project itself. Sometimes you will also find boxes like this one. These are meant to provide you with explanations and insights or hints that go beyond the mere structure of a thesis. 

I hope this template will help you do the best thesis ever, if not in the World, at least in your life.

\hfill Florian Daniel

\hfill October 12, 2017

\bigskip \noindent \emph{Disclaimer:} Sometimes I may make statements that are general, if not over-generalized, personal considerations, or give hints on how to do work or research. Be aware that these are just my own opinions and by no way represent official statements by Politecnico di Milano or its community of professors. If something goes wrong with your thesis or presentation, you cannot refer to these statements as a defense. You are the final responsible of what goes into your thesis and what not.

\medskip \noindent \emph{Acknowledgements:} The original template for this document was not created by me. I would love to acknowledge the real creator, but I actually do not know who it is. The template has been passed on to me by a former student, who also didn't know the exact origin of it. It was circulating among students. However, to the best of my knowledge at the time of writing, it seems that Marco D. Santambrogio and Matto Matteucci may have contributed at some point with considerations on structure and funny citations. Both were helpful and enjoyable when preparing this version of the template. I will be glad to add more precise acknowledgements if properly informed about the origins of this template.} 

\note{Supervisors and co-supervisors}{If the supervisor is internal to Politecnico di Milano (a professor or researcher), then on the first page use ``Supervisor'' plus the titles ``Prof.'' and ``Dr.'' for professors and researches, respectively. If the work was co-supervised by someone else, refer to him/her as the ``Co-supervisor.'' If the work was supervised by someone external to Politecnico di Milano, use ``External supervisor'' for the external supervisor plus ``Internal supervisor'' for the internal supervisor that mandatorily must co-supervise the work with the external supervisor. }

%----------------------------------------------------------------------------
%	PREAMBLE PAGES: ABSTRACT (inglese e italiano), EXECUTIVE SUMMARY
%----------------------------------------------------------------------------
\cleardoublepage
\vspace*{2cm}
\begin{flushright}
\itshape{Optionally, here goes the dedication.}
\end{flushright}
\thispagestyle{empty}  \cleardoublepage

\startpreamble
\setcounter{page}{1} % Set page counter to 1

% ABSTRACT IN ENGLISH
% ABSTRACT IN ENGLISH
\chapter*{Abstract} 
The abstract is a small summary of the thesis. It tells the reader in few words (up to one/one and a half page of total text) everything he/she needs to understand: 

\begin{itemize}
\item[\Square] the \emph{context} of the work (e.g., chatbots),
\item[\Square] the specific \emph{problem} approached by the thesis (e.g., the development of personal bots by non-programmers), 
\item[\Square] if applicable, clearly state the \emph{research questions} you would like to answer (e.g., ``is it possible to enable non-programmers to do X using A?''),
\item[\Square] the three/four \emph{core aspects of the proposed solution} (e.g., use pre-defined rules, use machine learning, assisted development, etc.), 
\item[\Square] the \emph{concrete outputs} produced by the thesis (e.g., a state of the art analysis, a conceptual/mathematical model, an application, middleware or API, an empirical study with/without users, etc.), and 
\item[\Square] the \emph{findings and conclusions} that one can draw from the evaluation of the approach (e.g., that under some very specific conditions non-programmers are indeed able to implement own chatbots effectively using the proposed technique).
\end{itemize}


\note{Checklists}{Now and there I propose checklists with items, such as the one just above this box. They are meant for you to check if you included all the content that is relevant and that should be included, in order to make your text complete. When reading your thesis, I will look for all these items.}

The abstract is also the part appearing in the record of your thesis inside POLITesi,
the Digital Archive of PhD and Master Theses (Laurea Magistrale) of Politecnico di Milano.
The abstract will be followed by a list of four to six keywords.

Keywords are a tool to help indexers and search engines to find relevant documents.
To be relevant and effective, keywords must be chosen carefully.
They should represent the content of your work and be specific to your field or sub-field.
Keywords may be a single word or two to four words. 
\\
\\
\textbf{Keywords:} here, the keywords, of your thesis % Keywords

\note{Writing style}{This is a M.Sc. thesis. It's neither Facebook nor Twitter nor an email. This is going to be an official document with legal value that will decide on the final mark of your yearlong university career and perhaps even on your future work perspectives. So, you surely don't want to be judged badly because of grammar errors, flawed/wrong vocabulary or superficial layout and/or text structure. It is a must that what you write is always \emph{correct} content- and language-wise (no false statements or claims, no language mistakes), \emph{readable} (no sentences that cannot be understood) and targeted at the \emph{average-skilled reader} (professors, but also your own colleagues).}

\note{Plagiarism and bibliography}{This is a M.Sc. thesis. It's neither Facebook nor Twitter nor an email. This is going to be an official document with legal value that will decide on the final mark of your yearlong university career  and perhaps even on your future work perspectives -- yes, I plagiarized myself here a little bit. So, you surely don't want to copy/paste material from scientific articles, online resources, books, and similar without adequately acknowledging the holders of the respective intellectual property rights. If you do so, it is a must that you properly \emph{cite} each source where you take text or inspiration from. It is fine to do so -- actually, citing someone is a compliment! -- but it becomes a crime if the source is not cited. Not only M.Sc. titles but also Ph.D. titles have been withdrawn for fraudulent ``reuse'' of others' intellectual property. As stated in the Code of Ethics and Conduct, Politecnico di Milano \textit{promotes the integrity of research, condemns manipulation and the infringement of intellectual property}, and gives opportunity to all those who carry out research activities to have an adequate training on ethical conduct and integrity while doing research.
Be aware that Politecnico di Milano, like most higher educational institutions that issue university degrees or scientific publishers, may use specialized software to automatically detect plagiarism.
%
Your thesis must contain a suitable Bibliography which lists all the sources consulted on developing the work.
The list of references is placed at the end of the manuscript after the chapter containing the conclusions.
You can use one of the many bibliographic packages available for \LaTeX, for example with the BibTeX package you can save the bibliographic references in a \emph{.bib} file.
To make a citation in your manuscript, you can use  BibTeX~\cite{bibtex} commands as follows:
%
\textit{here is how you cite bibliography entries: \cite{knuth74}, or multiple ones at once: \cite{knuth92,lamport94}}.
%
BibTeX will take care of generating the bibliography for you.
}



% ABSTRACT IN ITALIAN
\chapter*{Sommario}
Here goes the Italian translation of the abstract and keywords.
\\
\\
\textbf{Parole chiave:} qui, vanno, le parole chiave, della tesi % Keywords (italian)

%----------------------------------------------------------------------------
%	LIST OF CONTENTS/FIGURES/TABLES/SYMBOLS
%----------------------------------------------------------------------------

% TABLE OF CONTENTS
\thispagestyle{empty}
\tableofcontents % Table of contents 
\thispagestyle{empty}
\cleardoublepage

%-------------------------------------------------------------------------
%	THESIS MAIN TEXT
%-------------------------------------------------------------------------
% In the main text of your thesis you can write the chapters in two different ways:
%
%(1) As presented in this template you can write:
%    \chapter{Title of the chapter}
%    *body of the chapter*
%
%(2) You can write your chapter in a separated .tex file and then include it in the main file with the following command:
%    \chapter{Title of the chapter}
%    \input{chapter_file.tex}
%
% Especially for long thesis, we recommend you the second option.

\addtocontents{toc}{\vspace{2em}} % Add a gap in the Contents, for aesthetics
\mainmatter % Begin numeric (1,2,3...) page numbering

% --------------------------------------------------------------------------
% NUMBERED CHAPTERS % Regular chapters following
% --------------------------------------------------------------------------

\input{1_introduction}

\input{2_state_of_the_art}

\chapter{[Core contribution]: Goals and Requirements}
\label{capitolo3}
\thispagestyle{empty}


This chapter splits the problem that so far was still at a relatively abstract and intuitive level of understanding down into fine-grained sub/problems, which then lead to concrete action items to be approached throughout the thesis project. This is the chapter where you show your understanding of the \emph{problem}. As such, it is important, on the one hand, to show your competence and, on the other hand, to explain the reader what exactly you are going to work on.

\begin{itemize}
\item[\Square] Replace the ``[Core contribution]'' in the title of the chapter with the name of the core contribution of your thesis work. If, for example, your contribution is the design and evaluation of a modeling language for the modeling of crowdsourcing processes, you could use something like ``Modeling Crowdsourcing Processes: Goals and Requirements.''
\end{itemize}



\section{Concepts}
In the introduction, you already introduced the core terminology needed to understand the preliminary problem statement. Here you may want to provide more details and more terminology, as things now get more concrete and new concepts may be needed to explain what you are working on.

\begin{itemize}
\item[\Square] Provide all the \emph{definitions} of concepts that you need to explain your work and that you did not yet introduce in the introduction.
\item[\Square] For each new definition, don't forget to provide clear \emph{examples}. 
\end{itemize}



\section{Goals and Requirements}
Here you repeat the initial problem statement of Section \ref{sec:scenario} and possibly refine it using the refined terminology introduced just now. Solving the problem is the goal of your thesis. Clarify who you think is the target user or beneficiary of your work. Then reason about the goals, considering the context of your work, your competences, possible constraints imposed to the potential solution, etc. and identify a set of \emph{requirements} that you want to meet with your solution (by now, you should know about requirements from Software Engineering or other classes):

\begin{itemize}
\item[\Square] \emph{Functional requirements} (expected functionalities supported by the solution)
\item[\Square] \emph{Generic non-functional requirements} (expected performance/quality levels)
\item[\Square] \emph{Architectural requirements} (e.g., if your solution is to be integrated into an existing system)
\item[\Square] \emph{Technological requirements} (e.g., if your solution must use given technologies)
\end{itemize}

Try to be concrete and not too abstract. After this section, the reader should really understand what to expect from your thesis. Ideally, you (or the reader) should be able to use the list of identified requirements as a checklist to be checked in the end of this document and, again ideally, for each requirement it should be possible to decide (true/false) if it is met or not. This may ask for the definition of suitable metrics to measure satisfaction. However, here it's too early to talk about that; this will go into the evaluation chapter.


\section{[Background one]}
If your work builds on prior work or research, this is the place where you can introduce the necessary knowledge to the reader. For instance, if you work on business process modeling and it is your goal to develop an extension of the modeling language BPMN, here you provide the necessary background knowledge so that the reader will be able to follow your subsequent discussions on the matter. Be cautious to introduce all and only those concepts, constructs, tools, languages that you really need.

\section{[Background two]}
If your work builds on more than one prior work or research, add respective sections. For instance, if your extension of BPM is meant to leverage on crowdsourcing to perform work, here you provide the necessary background on crowdsourcing.

\input{4_approach}

\input{5_implementation_evaluation}

\input{6_conclusion_future_work}


%-------------------------------------------------------------------------
%	BIBLIOGRAPHY
%-------------------------------------------------------------------------

\addtocontents{toc}{\vspace{2em}} % Add a gap in the Contents, for aesthetics
\bibliography{Thesis_bibliography} % The references information are stored in the file named "Thesis_bibliography.bib"

%-------------------------------------------------------------------------
%	APPENDICES
%-------------------------------------------------------------------------

\cleardoublepage
\addtocontents{toc}{\vspace{2em}} % Add a gap in the Contents, for aesthetics
\appendix
\chapter{Appendix: How to use figures, tables, equations, algorithms, \ldots}
If you need to include an appendix to support the research in your thesis, you can place it at the end of the manuscript.
An appendix contains supplementary material which supplement the main results contained in the previous chapters.

\note{Figures and tables}{You are an engineer, and using figures (illustrations) and tables to better convey your ideas should be an obvious practice you should have learned throughout your university career. If not, it's time now. Use illustrations, screen shots, sketches, and so on to help the reader understand. Use tables to summarize complex text (for example, a profound analysis of the state of the art) or to format data in a readable fashion. Each time you use a figure or table, you must also (i) complement it with a so-called caption (a text right underneath or above it) to give it a title and a description and (ii) reference it from within the main text (never just place a figure somewhere without talking about it). If you use Latex, check your Latex documentation for how to use captions and references. In the following you can find basic examples and general guidelines.}

% ---------------------------------------------------------------------------

\subsection*{Figures}You can use \texttt{TikZ} for high-quality hand-made figures, or just include them with the command
\begin{verbatim}
\includegraphics[options]{filename.xxx}
\end{verbatim}
Here xxx is the correct format, e.g. \verb|.png|, \verb|.jpg|, \verb|.pdf|, \verb|.eps|, \dots.

\begin{figure}[H]
    \centering
    \includegraphics[width=0.3\textwidth]{logo_polimi_scritta.eps}
    \caption{Caption of the Figure to appear in the List of Figures.}
    \label{fig:quadtree}
\end{figure}

Thanks to the \texttt{\textbackslash subfloat} command, a single figure, such as Figure~\ref{fig:quadtree},
can contain multiple sub-figures with their own caption and label, e.g. Figure~\ref{fig:polimi_logo1} and Figure~\ref{fig:polimi_logo2}. 

\begin{figure}[H]
    \centering
    \subfloat[One PoliMi logo.\label{fig:polimi_logo1}]{
        \includegraphics[scale=0.5]{Images/logo_polimi_scritta.eps}
    }
    \quad
    \subfloat[Another one PoliMi logo.\label{fig:polimi_logo2}]{
        \includegraphics[scale=0.5]{Images/logo_polimi_scritta2.eps}
    }
    \caption[Shorter caption]{This is a very long caption you don't want to appear in the List of Figures.}
    \label{fig:quadtree2}
\end{figure}\textbf{}

Remember to add a tilde ($\sim$) between the word Figure and the \verb|\ref{}| command (e.g. \verb|Figure~\ref{fig:polimi_logo1}|), to avoid automatic line breaking before the generated reference. The same should hold for citations, tables, equations, etc.

% ---------------------------------------------------------------------------
\subsection*{Tables}Within the environments \texttt{table} and  \texttt{tabular} you can create very fancy tables as the one shown in Table~\ref{table:example}.
\begin{table}[H]
    \caption*{\textbf{Title of Table (optional)}}
    \centering 
    \begin{tabular}{|p{3em} c c c |}
    \hline
    \rowcolor{bluepoli!40} % comment this line to remove the color
     & \textbf{column 1} & \textbf{column 2} & \textbf{column 3} \T\B \\
    \hline \hline
    \textbf{row 1} & 1 & 2 & 3 \T\B \\
    \textbf{row 2} & $\alpha$ & $\beta$ & $\gamma$ \T\B\\
    \textbf{row 3} & alpha & beta & gamma \B\\
    \hline
    \end{tabular}
    \\[10pt]
    \caption{Caption of the Table to appear in the List of Tables.}
    \label{table:example}
\end{table}

You can also consider to highlight selected rows in order to make tables more readable. One example is presented in Table~\ref{table:exampleC}. 

\begin{table}[h]
\centering 
    \begin{tabular}{|p{3em} | c | c | c | c | c | c|}
    \hline
     & \textbf{column1} & \textbf{column2} & \textbf{column3} & \textbf{column4} & \textbf{column5} & \textbf{column6} \T\B \\
    \hline \hline
    \textbf{row1} & 1 & 2 & 3 & 4 & 5 & 6 \T\B\\
    \rowcolor{bluepoli!40}
    \textbf{row2} & a & b & c & d & e & f \T\B\\
    \textbf{row3} & $\alpha$ & $\beta$ & $\gamma$ & $\delta$ & $\phi$ & $\omega$ \T\B\\
    \textbf{row4} & alpha & beta & gamma & delta & phi & omega \B\\
    \hline
    \end{tabular}
    \\[10pt]
    \caption{Highlighting one row}
    \label{table:exampleC}
\end{table}

% ---------------------------------------------------------------------------

\subsection*{Equations}Maxwell's equations read:
\begin{subequations}
    \label{eq:maxwell}
    \begin{align}[left=\empheqlbrace]
    \nabla\cdot \bm{D} & = \rho, \label{eq:maxwell1} \\
    \nabla \times \bm{E} +  \frac{\partial \bm{B}}{\partial t} & = \bm{0}, \label{eq:maxwell2} \\
    \nabla\cdot \bm{B} & = 0, \label{eq:maxwell3} \\
    \nabla \times \bm{H} - \frac{\partial \bm{D}}{\partial t} &= \bm{J}. \label{eq:maxwell4}
    \end{align}
\end{subequations}

You can now reference Equation~\eqref{eq:maxwell}, as well as Equation~\eqref{eq:maxwell1} and Equation~\eqref{eq:maxwell3}, using \verb|\eqref|.

Equations~\eqref{eq:maxwell_multilabels1}, \eqref{eq:maxwell_multilabels2}, \eqref{eq:maxwell_multilabels3}, and \eqref{eq:maxwell_multilabels4} show again Maxwell's equations without brace:
\begin{align}
    \nabla\cdot \bm{D} & = \rho, \label{eq:maxwell_multilabels1} \\
    \nabla \times \bm{E} +  \frac{\partial \bm{B}}{\partial t} &= \bm{0}, \label{eq:maxwell_multilabels2} \\
    \nabla\cdot \bm{B} & = 0, \label{eq:maxwell_multilabels3} \\
    \nabla \times \bm{H} - \frac{\partial \bm{D}}{\partial t} &= \bm{J} \label{eq:maxwell_multilabels4}.
\end{align}

Equation~\eqref{eq:maxwell_singlelabel} is the same as before,
but with just one label:
\begin{equation}
    \label{eq:maxwell_singlelabel}
    \left\{
    \begin{aligned}
    \nabla\cdot \bm{D} & = \rho, \\
    \nabla \times \bm{E} +  \frac{\partial \bm{B}}{\partial t} &= \bm{0},\\
    \nabla\cdot \bm{B} & = 0, \\
    \nabla \times \bm{H} - \frac{\partial \bm{D}}{\partial t} &= \bm{J}.
    \end{aligned}
    \right.
\end{equation}

% ---------------------------------------------------------------------------
\subsection*{Algorithms}
Algorithms and pseudo-code can be written in \LaTeX{} with the \texttt{algorithm} and \texttt{algorithmic} packages.
An example is shown in Algorithm~\ref{alg:var}.
\begin{algorithm}[H]
    \label{alg:example}
    \caption{Name of the Algorithm}
    \label{alg:var}
    \label{protocol1}
    \begin{algorithmic}[1]
    \STATE Initial instructions
    \FOR{$condition$}
    \STATE{Some instructions}
    \IF{$condition$}
    \STATE{Some other instructions}
    \ENDIF
    \ENDFOR
    \WHILE{$condition$}
    \STATE{Some further instructions}
    \ENDWHILE
    \STATE Final instructions
    \end{algorithmic}
\end{algorithm} 


% ---------------------------------------------------------------------------
\bigskip
\subsection*{Theorems}
Theorems have to be formatted as:
\begin{theorem}
\label{a_theorem}
Write here your theorem. 
\end{theorem}
\textit{Proof.} If useful you can report here the proof.

\subsection*{Propositions}
Propositions have to be formatted as:
\begin{proposition}
Write here your proposition.
\end{proposition}

\subsection*{Lists}
This is an itemized lists:
\begin{itemize}
    \item first item;
    \item second item.
\end{itemize}

This is a numbered lists:
\begin{enumerate}
    \item first item;
    \item second item.
\end{enumerate}


\chapter{Appendix B}
If necessary, you can include another appendix to better organize the presentation of supplementary material.
 %% appendix: figures, tables, etc.

% LIST OF FIGURES
\listoffigures
This list is automatically generated using the command \verb|\listoffigures|. You can decide together with your supervisor to move this list after the Table of Content. Note that the caption of Figure~\ref{fig:quadtree2} written here is different from the one under the actual figure.

% LIST OF TABLES
\listoftables
This list is automatically generated using the command \verb|\listoftables|. You can decide together with your supervisor to move this list after the Table of Content.

% % LIST OF SYMBOLS
% % Write out the List of Symbols in this page
% \chapter*{List of Symbols} % You have to include a chapter for your list of symbols (
% \begin{table}[H]
%     \centering
%     \begin{tabular}{lll}
%         \textbf{Variable} & \textbf{Description} & \textbf{SI unit} \\\hline\\[-9px]
%         $\bm{u}$ & solid displacement & m \\[2px]
%         $\bm{u}_f$ & fluid displacement & m \\[2px]
%     \end{tabular}
% \end{table}

%-------------------------------------------------------------------------
%	ACKNOWLEDGEMENTS
%-------------------------------------------------------------------------
\chapter*{Acknowledgements}
If you would like to thank somebody for given support, this is the right place to do so. Generally, trying to  fit everything in a single page is a good idea. If you want, you can ask your supervisor if you can move this section right before the Table of Contents, after the Abstract/Sommario.

\cleardoublepage

\end{document}
